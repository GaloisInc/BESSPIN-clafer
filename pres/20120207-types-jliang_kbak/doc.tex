\documentclass[table,15pt,t]{beamer}

\usepackage{fontspec}
\usepackage{xunicode} %Unicode extras!
\usepackage{xltxtra}  %Fixes
\usepackage{relsize}  %relative font sizing commands
\usepackage{booktabs}
\usepackage{eulervm} %mathfonts
\usepackage{tikz}
\usepackage{listings}
\usepackage{graphicx}
\usepackage{tabularx}
\usepackage{alltt}
\usepackage{ifthen,calc}
\usepackage{mathpartir}

\usetheme{Sleek}

\usetikzlibrary{arrows,shapes,shapes.arrows,positioning}
\usetikzlibrary{chains}

\tikzstyle{ds} = [draw, thick, color=oxygenblue, inner sep=0pt, drop shadow]

\setsansfont[Mapping=tex-text]{Trade Gothic LT Std}

\setlength{\fboxsep}{2mm}
\newcommand{\sem}[1]{\ensuremath{[\![#1]\!]}}

\title{Clafer Type System and Attributes}
\author{Jimmy Liang and Kacper Bak}
\institute{\small Generative Software Development Lab\\University of Waterloo}
\date{GSD Lab Workshop. February 7, 2012}

\newcommand{\vmiddle}[1]{
  \vspace{\stretch{1}}
  #1
  \vspace{\stretch{1}}
}

\newcommand{\interframe}[1]{
\begin{frame}{}
\vmiddle{\hmiddle{\Huge #1}}
\end{frame}
}

\newcommand{\mlist}[1]{
\vmiddle{
  \begin{list}{}{}
    #1
  \end{list}
  }
}

\newcommand{\hmiddle}[1]{
  \begin{center}#1\end{center}
}

% To make the legend for icons in class diagrams
\newcommand{\icon}[1]{\textcircled{\texttt{\textbf{\textit{#1}}}}}

\lstdefinelanguage{Clafer}
{morekeywords={abstract, else, in, no, opt, xor, all, enum, lone, not, or, disj, extends, mux, one, some},
sensitive=true,
morecomment=[l][\footnotesize\itshape]{--},
morecomment=[s][\small\itshape]{{-}{-}},
%basicstyle=\footnotesize,
tabsize=4,
columns=fullflexible,
literate={->}{{$\to$ }}1 {^}{{$\mspace{-3mu}\widehat{\quad}\mspace{-3mu}$}}1
 {<}{$<$ }2 {>}{$>$ }2 {>=}{$\geq$ }2 {<=}{$\leq$ }2
 {<:}{{$<\mspace{-3mu}:$}}2 {:>}{{$:\mspace{-3mu}>$}}2
 {<=>}{{$\Leftrightarrow$ }}2
 {=>}{{$\Rightarrow$ }}2 {+}{$+$ }2 {++}{{$+\mspace{-8mu}+$ }}2
 {\~}{{$\mspace{-3mu}\widetilde{\quad}\mspace{-3mu}$}}1
 {!=}{$\neq$ }2 {*}{${}^{\ast}$}1
 {\#}{$\#$}1
 {\~}{$\neg$}1
 {+}{$+$ }1
}

\lstset{frame=single,basicstyle=\small,language=Clafer,numbers=none}

\begin{document}

\begin{frame}[plain]
  \vmiddle{\titlepage}
\end{frame}

\interframe{Recent Progress}

% 5 min for intro and recent progress - Kacper
\begin{frame}{Recent Progress}
  \mlist{
  \item Many people involved
  \item Language
  \item Applications and ecosystem
  \item Documentation and promotion
  }
\end{frame}

\begin{frame}{Recent Progress: Language}
  \mlist{
  \item Defaults (Leo)
  \item Type system (Jimmy)
  \item Intermediate representation - XML (Jimmy)
  \item Translator (Jimmy, Kacper)
  \item Test suite (Michal, Kacper)
  \item Visualization: CVL and Class-like (Seyi)
  \item Natural language syntax proposal (Michal)
  }
\end{frame}

\begin{frame}{Recent Progress: Applications}
  \mlist{
  \item Lightweight methodology (Krzysztof, Michal)
  \item Multiobjective optimization (Bo, Derek)
  \item Usability empirical evaluation (Dina)
  \item Experiments with bug tracking (Rafael)
  \item Future: financial domain (Marko), security (with Mahesh)
  }
\end{frame}

\begin{frame}{Recent Progress: Documentation}
  \mlist{
  \item Tutorial (Michal)
  \item Clafer wiki: github.com/gsdlab/clafer/wiki/ (Michal, Kacper)
  \item Website: clafer.org (Michal, Kacper)
  }
\end{frame}

% 15min for type system - Jimmy; examples, typing rules, a few words about implementation
% motivate the need for type system
% show examples (clafer code and typing rules)
% more formal stuff
% implementation
\interframe{Type System}

\begin{frame}{Motivation}
\vmiddle{Why does Clafer need a type system?}
\end{frame}

\begin{frame}[fragile,c]{Type check}
Free and automatic sanity check.
\vfill \begin{lstlisting}
abstract Y
    x : string
    y : int
    [x + y = 3]
\end{lstlisting}
Clafer reports that ``+'' cannot be applied on $x$ and $y$.
\end{frame}

\begin{frame}[fragile,c]{Semantics}
The type of an expression affects the semantics (output Alloy code). Also, needed for code optimization.
\vfill \begin{lstlisting}
computer
    dualCpu ?
        speed : integer
    totalSpeed : integer
\end{lstlisting}
What does the expression $speed$ mean?
\end{frame}

\begin{frame}[fragile,c]{Two cases}
\begin{lstlisting}
computer
    dualCpu ?
        speed : integer
    extraSpeed : integer
    [extraSpeed = (speed => speed else 0)]
\end{lstlisting}
The expression $speed$ refers 1) presence of clafer, 2) to its integer value.
\end{frame}

\begin{frame}[fragile,c]{Semantics OO-analogy}
\lstset{language=Java}
\begin{lstlisting}
class MemberAge {
    int value;
}
class CentenarianClub {
    MemberAge[] memberAge;
}
\end{lstlisting}
The expression $memberAge$ can refer to
\begin{itemize}
\item the $MemberAge$ object(s)
\item or its $value$ field
\end{itemize}

The inferred type of the expression implicitly determines which of the two case applies.
\end{frame}

\begin{frame}[fragile,c]{Type system}
A Clafer model consists of two parts.
\begin{itemize}
\item Clafer definitions
\item Constraints
\end{itemize}

\vfill The type system performs two tasks in parallel.
\begin{itemize}
\item Type check the expressions in the constraints.
\item Infer the types of expressions in the constraints.
\end{itemize}
\end{frame}

\begin{frame}[fragile,c]{Notation \& Definition 1}
\begin{definition}
``$::$'' is shorthand for ``is type''.
\end{definition}
example: ``$x::integer$'' is read as ``$x$ is type $integer$''.

\begin{definition}
``$\vdash$'' is shorthand for ``entails''.
\end{definition}

example: ``$\Gamma\vdash x::integer$'' is read as ``$\Gamma$ entails $x::integer$''.

Sometimes it is clearer to read it as ``$x::integer$ given $\Gamma$''.
\end{frame}


\begin{frame}[fragile,c]{Notation \& Definition 2}
\begin{definition}
The letter ``$x$'' is a Clafer reference.
\end{definition}

In the expression below, ``$speed$'' is a Clafer reference.
\begin{lstlisting}
[speed > 80]
\end{lstlisting}
\end{frame}

\begin{frame}[fragile,c]{Notation \& Definition 3}

\begin{definition}
The letters ``$E,F,G$'' are expressions.
\end{definition}

2 leaf expressions: ``$speed$'' and ``$80$''.

1 super expression: ``$speed > 80$''.
\begin{lstlisting}
[speed > 80]
\end{lstlisting}

\vfill Less frequent notations will be explained as they come.
\end{frame}

\begin{frame}[fragile,c]{Notation \& Definition 4}

\begin{definition}
A type environment ($\Gamma$) is a mapping from Clafer definition to the type of its value.
\end{definition}

\vfill \begin{lstlisting}
abstract Y : string
    a : integer
    b

X : Y
\end{lstlisting}

\begin{equation*}
\Gamma = \{ Y::string,\quad a::integer,\quad b::clafer,\quad X::string \}
\end{equation*}
\end{frame}

\begin{frame}[fragile,c]{Type rule}
\begin{equation*}
\inferrule* [Left=name of rule] {statement A \\ statement B} {statement C}
\end{equation*}

\vfill If $A$ and $B$ holds then $C$ follows.

\vfill The type system is specified through a series of type rules.
\end{frame}

\begin{frame}[fragile,c]{Clafer type rule 1}
\begin{equation*}
\inferrule* [Left=intconst] {\quad}{\Gamma\vdash\mathbb{INTEGER}::integer}
\end{equation*}
\end{frame}

\begin{frame}[fragile,c]{Clafer type rule 2}
\begin{equation*}
\inferrule* [Left=eq] {\Gamma\vdash E::\tau \\ \Gamma\vdash F::\tau}{\Gamma\vdash E = F::boolean}
\end{equation*}
\end{frame}

\begin{frame}[fragile,c]{Clafer type rule 3}
Can we prove that the following model passes type checking?
What is the type of each expression?

\begin{lstlisting}
abstract Y
    [0 = 1]
\end{lstlisting}

\begin{equation*}
\Gamma = \{Y::clafer\}
\end{equation*}
\end{frame}

\begin{frame}[fragile,c]{Clafer type rule 4}
\begin{equation*}
\Gamma = \{Y::clafer\}
\end{equation*}

\begin{proof}
\begin{equation*}
\inferrule* [Left=eq] {
  \inferrule* [Left=intconst]{\quad}{\Gamma\vdash\ 0::integer} \\ \hspace{40pt} 
  \inferrule* [Left=intconst]{\quad}{\Gamma\vdash\ 1::integer}}
{\Gamma\vdash 0 = 1::boolean}
\end{equation*}
\end{proof}
\end{frame}

\begin{frame}[fragile,c]{Clafer type rule 5}
\begin{equation*}
\inferrule* [Left=value] {(x::\tau)\in\Gamma}{\Gamma\vdash x::\tau}
\end{equation*}
\end{frame}

\begin{frame}[fragile,c]{Clafer type rule 6}

 Prove that the following model is type correct.

\begin{lstlisting}
abstract Y
    a : integer
    [a = 1]
\end{lstlisting}

\begin{equation*}
\Gamma = \{Y::clafer,\quad a::integer\}
\end{equation*}
\end{frame}

\begin{frame}[fragile,c]{Clafer type rule 7}
\begin{equation*}
\Gamma = \{Y::clafer,\quad a::integer\}
\end{equation*}

\begin{proof}
\begin{equation*}
\inferrule* [Left=eq] {
  \inferrule* [Left=value]{(a::integer)\in\Gamma}{\Gamma\vdash\ a::integer} \\ \hspace{40pt} 
  \inferrule* [Left=intconst]{\quad}{\Gamma\vdash\ 1::integer}}
{\Gamma\vdash a = 1::boolean}
\end{equation*}
\end{proof}
\end{frame}

\begin{frame}[fragile,c]{Clafer type rule 8}
\begin{equation*}
\inferrule* [Left=clafer] {\quad}{\Gamma\vdash x::clafer}
\end{equation*}
\end{frame}

\begin{frame}[fragile,c]{Clafer type rule 9}

 Prove that the following model is type correct.

\begin{lstlisting}
abstract Y
    a : integer
    b
    [a = b]
\end{lstlisting}

\begin{equation*}
\Gamma = \{Y::clafer,\quad a::integer,\quad b::clafer\}
\end{equation*}
\end{frame}

\begin{frame}[fragile,c]{Clafer type rule 10}
\begin{equation*}
\Gamma = \{Y::clafer,\quad a::integer,\quad b::clafer\}
\end{equation*}

\begin{proof}
\begin{equation*}
\inferrule* [Left=eq] {
  \inferrule* [Left=clafer]{\quad}{\Gamma\vdash\ a::clafer} \\ \hspace{40pt} 
  \inferrule* [Left=clafer]{\quad}{\Gamma\vdash\ b::clafer}}
{\Gamma\vdash a = b::boolean}
\end{equation*}
\end{proof}
\end{frame}

\begin{frame}[fragile,c]{Clafer type rule precedence}
\begin{lstlisting}
abstract Y
    a : integer
    b : integer
    [a = b]
\end{lstlisting}

\vfill Integer equality or clafer equality?
\end{frame}

\begin{frame}[fragile,c]{Clafer type rule casting 1}
\begin{lstlisting}
abstract Y
    a : real
    b : integer
    [a = b]
\end{lstlisting}
\end{frame}

\begin{frame}[fragile,c]{Clafer type rule casting 2}
\begin{equation*}
\inferrule* [Left=eqcast1] {\Gamma\vdash E::real \\ \Gamma\vdash F::integer}{\Gamma\vdash E = F::boolean}
\end{equation*}

\begin{equation*}
\inferrule* [Left=eqcast2] {\Gamma\vdash E::integer \\ \Gamma\vdash F::real}{\Gamma\vdash E = F::boolean}
\end{equation*}
\end{frame}


% 15 minutes
\interframe{Attributes}

\begin{frame}[fragile,c]{Primitive Types}
\begin{lstlisting}
Version : int *
[Version + 2 > 0]
\end{lstlisting}\pause
\begin{list}{}{}
    \item What is the value of Version?\pause
    \item Is the constraint satisfied? When?\pause
    \item Is the meaning of ``+'' purely arithmetic?\pause
    \item Does ``+'' handle sets?
\end{list}
\end{frame}

\interframe{Semantic Variants}

\begin{frame}[fragile,c]{Sum}
\begin{lstlisting}
Version : int *
[Version + 2 > 0]
\end{lstlisting}
\begin{list}{}{}
    \item Semantics: [sum(Version) + 2 > 0]\pause
    \item Version 0: returns 0. OK
    \item Version 1: OK
    \item Version +: Set sum. Confusing
\end{list}\pause
\begin{lstlisting}
abstract comp
  version : int
[comp.version = 1]
\end{lstlisting}\pause
Used by Alloy and CDL
\end{frame}

\begin{frame}[fragile,c]{Universal Quantification}
\begin{lstlisting}
Version : int *
[Version + 2 > 0]
\end{lstlisting}
\begin{list}{}{}
    \item Semantics: [all a : Version | a + 2 > 0]\pause
    \item Version 0: ignores constraint
    \item Version 1: OK
    \item Version +: OK
\end{list}
\end{frame}

\begin{frame}[fragile,c]{Enforced Presence}
\begin{lstlisting}
Version : int *
[Version + 2 > 0]
\end{lstlisting}
\begin{list}{}{}
    \item Semantics: [some Version \&\& Version + 2 > 0]\pause
    \item Version 0: constraint unsatisfied (as in OCL)
    \item Version 1: OK
    \item Version +: unclear. Sum or universal quantification.
\end{list}
\end{frame}

\begin{frame}[fragile,c]{Explicit Guards}
\begin{lstlisting}
Version : int *
[Version + 2 > 0]
\end{lstlisting}
\begin{list}{}{}
    \item Semantics: [(Version ? Version : 0) + 2 > 0]\pause
    \item Version 0: OK
    \item Version 1: OK
    \item Version +: unclear. Sum or universal quantification.
\end{list}
\end{frame}

\begin{frame}[fragile,c]{Vector Operations}
\begin{lstlisting}
Version : int *
[Version + 2 > 0]
\end{lstlisting}
\begin{list}{}{}
    \item Semantics: [Version(0) + 2 > 0, Version(1) + 2 > 0, ...]\pause
    \item Version 0: unclear
    \item Version 1: OK
    \item Version +: OK, but complex.
\end{list}
\end{frame}

\interframe{No good semantics!}

\begin{frame}[fragile,c]{Proposed Solution}
\begin{lstlisting}
Version : int *
[Version + 2 > 0]
\end{lstlisting}
\begin{list}{}{}
    \item Use the sum semantics. Covers 0 and 1 cardinality.
    \item For higher cardinalities, enforce using quantifiers.
    \item Complex but makes intuitive sense.
    \item Discussion: github.com/gsdlab/clafer/wiki/Experimental:-Attributes
\end{list}
\end{frame}

\interframe{Conclusion}

\begin{frame}{Conclusion}
 \mlist{
    \item Clafer entered new evolution phase
    \item Type system clarifies semantics
    \item Experiments with design choices
    \item Evaluation needed
    \item Upcoming applications
 }
\end{frame}

\interframe{Thanks for listening!}

\interframe{Questions?\\[1cm]\normalsize{\textsf{clafer.org}}}

\end{document}
